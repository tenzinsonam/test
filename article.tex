\documentclass[12pt, a4paper]{article}
\usepackage[utf8]{inputenc}

\title{ENG 124 \\
Variations in Language in our daily Lives}
\author{Sonam Tenzin \\
150724}
\date{28 March 2017}

\begin{document}
\maketitle
\newpage
Language is the method of human communication, either spoken or written, consisting of the use of words in a structured and conventional way, as described in \textit{Wikipedia}. Thus it is the entire structure by which humans communicate with each other and this structure comprises of elaborate rules. These rules help in making \textit{a language} standard, which is very important as without them the goal of the structure will be defeated. It will become very difficult for people to understand each other without such structure.\\ \par 

 In our daily lives, we see ourselves executing this structure in quite a lot of ways which defined largely by our environment or the external elements. One of these elements is the person or people with whom we are communicating. These include friends, family, teachers, etc. We rarely notice the changes which occur in our speaking when talking to different people. These changes range from different languages to different vocabulary and many times different tones. For example, the way we talk in our home and with our friends is very different. The words which may seem eligible in our talk with friends may lose its meaning at home, even when in the same language. It is due to the fact that the concept which the word represents may not exist or might represent a different concept.  \\ \par 
 
 Another example is the constant language selection in case of multilingual people that their brain does which they barely seem to notice. This is due to the fact that after speaking languages for a long period, it becomes natural activity for them. Such as the case when speaking to a friend about some certain scientific concept. It can be clearly observed that we will definitely speak some English words no matter whichever language we were earlier conversing in and this change is rarely visible to us. By this example we can also claim that all the languages have their own characteristics which are unique to them.\\ \par
 
 The language variations can be also observed when we are communicating in a formal or in an informal way. The formal ones certainly have a bit stern rules compared to casual talks. For example, the manner of speaking varies widely when speaking to a professor and to that a friend. The change can also be observed between a formal and an informal mail. We might not notice but we put some restrictions on the words when we are speaking formally. Also in case of formal writings we tend to make it a fact to make them as concise as possible, so not waste the reader's time with unnecessary information whereas in the other case their is no such limit. \\ \par
 
 Social classes also seem to play an important role in these variations. These classes are defined on the basis of income. For example in an office, it expected of employees to talk in a respectful manner to his employer while there is no such obligation when talking to people of the same class or the working class. Although, this example is set in an office but it is pretty much true for everywhere in the world. The employees can communicate in any manner that they seem fine with, there is no such restriction as in the case of boss. Their tone and attitude might also vary when talking with the two, clearly representing the variation caused by social class \\ \par
 
 On broadening this situation of variation in communication, the electronic media additionally causes language variation. The language in emails is far unique in relation to the one in chats and forums. The last likewise fuses visual prompts, for example, emoticons, stickers and so on for communicating a few feelings viably, which are barely utilized as a part of mails. The mails differ in the level of formality and range from totally formal to an easygoing mail sent to a companion. Electronic communication  has tragically deterred the noteworthiness of stringent syntax principles and vocabulary in casual discussions. \\ \par
 
 Linguistic variation is likewise in view of the age gathering of speaker and audience in oral correspondence. Youthful and close gatherings regularly utilize their slang, which shapes a solid in-gathering and realizes a feeling of character among them. A similar arrangement of individuals when conversing with an elderly individual show an essence of gravity and care and cease from slang. This is an appearance of the way individuals suit their language towards the style of the other person . For example, individuals conversing with outsiders of various social classes utilize distinctive style and utilization of words. The linguistic parameters tone, disposition, certainty rely on upon the audience. An receptionist uses distinctive assortment of language from a neighborhood shop seller or that from a salesman.\\ \par
 
 




\end{document}

